% Basic setup
\documentclass[12pt,a4paper]{article} % Defines the document type as an article with 12pt font size on A4 paper.

% Importing packages
\usepackage[french]{babel} % Sets the document language to English, adjusting hyphenation and language-specific typographic rules.
\usepackage[lmargin=2.5cm,rmargin=2.5cm,tmargin=2.5cm,bmargin=2.5cm]{geometry} % Sets custom page margins: left/right 2.5cm, top 2.5cm, bottom 2.5cm.

% Loading packages
\usepackage{hyperref}    % Enables hyperlinks for references, URLs, and citations.
\usepackage{xcolor}      % Provides tools for defining and using colors.
\usepackage{graphicx}    % Allows inclusion of images and graphics.
\usepackage{caption}     % Customizes captions for figures and tables.
\usepackage{subcaption}  % Supports subfigures and subcaptions within figures.
\usepackage{minted}      % Enables syntax highlighting for code listings.
\usepackage[T1]{fontenc} % Ensures proper font encoding, important for correct character rendering. Don't touch this.
\usepackage{setspace}    % Provides control over line spacing.
\usepackage{csquotes}    % Improves handling of quotations.
\usepackage{longtable,booktabs,array} % Packages for advanced table formatting.

\setstretch{1.5} % Sets line spacing

\definecolor{LightGray}{gray}{0.9}  % Defines a custom color 'LightGray' with 90% gray, used for the code block background.
\hypersetup{                          % Configures hyperlink colors and behavior.
  colorlinks=true,                  % Enable colored links instead of boxes.
  linkcolor={blue},                 % Sets link color to blue.
  filecolor={maroon},               % Sets file link color to maroon.
  citecolor={blue},                 % Sets citation link color to blue.
  urlcolor={blue}}                  % Sets URL link color to blue.

\title{RaikoMusics Project Rapport}      % Sets the document title.
\makeatletter
\providecommand{\subtitle}[1]{%      % Custom command to add a subtitle.
  \apptocmd{\@title}{\par {\large #1 \par}}{}{}
}

\makeatother
\subtitle{Une application de streaming audio simple et rapide}  % Sets the document subtitle.
\author{Antoine Torrez Suzano} % Sets the author's name.
\date{\today}                       % Sets the document date to the current date.

% From this point, the preamble ends and the actual content of the document starts.
\begin{document}
\pagenumbering{gobble} % Stops counting the pages from this point until changed again.
\maketitle
\begin{abstract}
Ce document permet d'avoir un aperçu du project Raiko Musics Project. Raiko Musics est une application de streaming de musique simple et rapide qui permet à ses utilisateur d'écouter et de publier leur propre musique. Le projet à été développé avec l'objectif d'apprendre et d'approfondir les connsaissances sur de nombreuse technologie incluant : la streaming Audio, les pipeline CI/CD avec GitHub Actions, et l'interaction entre plusieurs conteneurs Docker. Ce document explique de manière détaillée les objectifs, l'architecture, les fonctionnalités et les technologie utilisées durant le développement du projet.
\end{abstract}
\begin{center}
    \vfill
    \begin{figure}[h!]
        \centering
        % You can add your own logos here if you want
        % \begin{subfigure}{.3\textwidth}
        %   \centering
        %   \includegraphics[width=.8\linewidth]{./assets/uni-basel-logo-en.png}
        % \end{subfigure}%
        % \begin{subfigure}{.3\textwidth}
        %   \centering
        %   \includegraphics[width=.8\linewidth]{./assets/dhlab-logo-black.png}
        % \end{subfigure}
        \end{figure}
        \setcounter{figure}{0}
        
    Rapport Projet\\
    Raiko Musics Project
\end{center}
\newpage
\renewcommand*\contentsname{Table des matières} % This controls the title of your table of contents.
{
\hypersetup{linkcolor=}
\setcounter{tocdepth}{5} % Sets the maximum sublevel to be displayed within the table of contents.
\tableofcontents
}
\newpage
\pagenumbering{arabic}\setstretch{1.5} % Overwrites the previous command, pages are counted as normal from this point.


\section{Introduction}

RaikoMusics est une légère application de streaming musical conçue pour sa simplicitée et sa rapidité. Elle permet aux utilisateur d'écouter de la musique directement et aux artistes de publier leur musique de façon intuitive, le but principal de RaikoMusics est d’être un prétexte à l'apprentissage de divers technologies qui concerne le DevOps.

La première version stable étant désormais en ligne , le projet sert aujourd’hui de fondation solide pour de multiples améliorations futures, tel que la conversion du streaming audio (de "HTTP range request" à du HLS), implémenter la gestion d'utilisateurs, la création de multiples playlists, le développement d'une algorithme de recommandation et bien plus encore !

Le projet est sous la \href{https://opensource.org/licenses/MIT}{Licence MIT}.

\section{Objectifs du Projet}

Les buts principal du projet Raiko Musics Project sont divisé en 3 catégories

\begin{itemize}
    \item \textbf{Apprentissage:} Gagner de l’expérience et apprendre de nouvelles technologies
    \item \textbf{Fondation Solide:} Avoir une application avec une fondation solide permettant de futurs amélioration a développer
    \item \textbf{Rapidité et simplicité:} Offrir une application avec un minimal d'option et de complexité pour écouter, publier et supprimer de la musique.

\end{itemize}

\section{Fonctionnalités et critère d'acceptation}

Pour considérer la version initial du projet comme terminé, les critères suivant doivent être satisfaits:

\begin{itemize}
    \item L'utilisateur peut écouter de la musique provenant du serveur.
    \item L'utilisateur peut publier de la musique.
    \item L'application est automatiquement mise à jour avec un push sur la branche main (Pipeline CI/CD)
    \item L'application est sous forme exécutable et accessible via internet
\end{itemize}

\section{Architecture et Technologies}

Raiko Musics Project est 

\subsection{Client-side}
The client is a desktop application for Windows, built using:
\begin{itemize}
    \item \textbf{Electron.js:} A framework for creating native applications with web technologies like JavaScript, HTML, and CSS.
    \item \textbf{React:} A JavaScript library for building user interfaces.
\end{itemize}

The user interface consists of a single playlist view, a music library, and a music player with basic controls (play/pause, next/previous track).

\subsection{Server-side}
The server-side is composed of three main services, all managed by Docker:
\begin{itemize}
    \item \textbf{Nginx Server:} Used as a reverse proxy and for handling audio streaming via HTTP range requests.
    \item \textbf{Node.js API Server:} An API built with Node.js and Express.js to handle music uploads and other interactions.
    \item \textbf{Music Storage:} A Docker volume is used to store the music files.
\end{itemize}
The entire server infrastructure is deployed on a RockyLinux 9.6 VM.

\subsection{CI/CD Pipeline}
The project uses a CI/CD pipeline built with \textbf{GitHub Actions}. This pipeline automates the process of building, testing, and deploying the application and server whenever changes are pushed to the main branch of the GitHub repository.

\section{Future Improvements}
Once the core functionalities are stable, several enhancements are planned for future versions of RaikoMusics:
\begin{itemize}
    \item \textbf{HLS Streaming:} Transition from HTTP range requests to HTTP Live Streaming (HLS) for better performance and adaptability.
    \item \textbf{User Management:} Implement user accounts with features like personalized playlists and listening history.
    \item \textbf{Enhanced Music Discovery:} Introduce features like multiple playlists, a recommendation algorithm, and advanced search capabilities.
    \item \textbf{Cross-platform Support:} Extend the application to other operating systems like macOS and Linux.
\end{itemize}


\end{document}